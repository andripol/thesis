\chapter{FLYCLIENT UNDER VELVET FORK}
In the FlyClient paper~\cite{flyclient} a velvet fork is suggested for the deployment of the protocol as-is, followed by a short argument for its respective security. In this document we describe an explicit attack against the FlyClient protocol under velvet fork deployment. This is essentially a kind of ``Chainsewing Attack'', a class of attacks that we have already described in our work on NIPoPoWs under velvet fork conditions.

\begin{section}{The FlyClient Protocol}
	The FlyClient protocol suggests that block headers additionaly include an MMR root of all the blocks in the chain. The protocol uses this root hash in multiple ways, both for chain synchronization and specific block queries. Consider a block $b$ which is appended to the chain $\chain$ at height $h_b$: 
	\begin{itemize}
		\item the prover generates a merkle inclusion proof $\Pi_b$ for the existence of $b$ at height $h_b$ in $\chain$ with respect to the MMR root included in the \emph{head} or \emph{tip} of the chain $\chain[-1]$
		\item the verifier receives the merkle root of the chain from a prover and an inclusion proof $\Pi_b$ for block $b$. He also generates from $\Pi_b$ the root of the MMR subtree of all blocks in $\chain$ from genesis up to $\chain[h_b - 1]$ and verifies that it is equal to the merkle root included in the header of block $b$.
	\end{itemize}
	The above proofs are produced with respect to the MMR root included in $\chain[-1]$.

	\vspace{2mm}
	\noindent
	A high level description of the FlyClient is as follows. Suppose that  the verifier, a superlight client, asks to synchronize to the current longest valid chain.  Suppose that he receives different proofs from two provers. Each prover sends (the header) of the last block in the chain, $\chain[-1]$, and a claim for the number of blocks in his chain, $\lvert \chain \rvert$. If both proofs are valid, then the one claiming the greater block count is selected. The validity check of a proof goes as follows. 
	The verifier has received $\chain[-1]$, $\lvert \chain \rvert$ and queries $k$ random block headers from each prover based on a specific probabilistic sampling algorithm. For each queried block $B_i$ the prover sends the header of $B_i$ along with an MMR subtree inclusion proof $\Pi_{B_i}$ that $B_i$ is the $i_\text{th}$ block in the chain. The verifier also checks that $B_i$ is normally mined on the same chain as $\chain[-1]$ by verifying that the root included in $B_i$ is the MMR root of the first ($\lvert \chain \rvert - 1$) blocks' subtree. If the $k$ random sampled blocks successfully pass through this verification procedure then the proof is considered valid, otherwise the proof is rejected by the verifier. 
	The chain synchronization protocol is given in Algorithm~\ref{alg:flyclient_suffix_protocol}.

	\begin{algorithm}[h]
		\caption{\label{alg:flyclient_suffix_protocol}FlyClient protocol~\cite{flyclient}}
		A client (i.e the Verifier) performs the following steps speaking with two provers who want to convince him that they hold a valid chain of length $n+1$. At least one of the provers is honest. If the provers claim different length for their claims then the longer chain is checked first.
		\begin{enumerate}
			\item The provers send to the verifier the last block header in their chain. Each header includes the root of an MMR created over the first $n$ blocks of the corresponding chain. 
			\item The verifier queries $k$ random block headers from each prover based on the described optimal probabilistic sampling algorithm.
			\item For each queried block, $B_i$, te prover sends the header of $B_i$ along with an MMR proof $\Pi_{B_i \in \chain}$ that $B_i$ is the $i-$th block in the chain.
			\item The client performs the following checks for each block $B_i$ according to Algorithm  and rejects the proof if any checks fail
			\item The client rejects the proof if any checks fail
			\item Otherwise, the client accepts $\chain$ as the valid chain
		\end{enumerate}
	\end{algorithm}

	\begin{algorithm}[h!]
		\caption{\label{alg:flyclient_iinfix_protocol}Prover/Verifier protocol for a single query~\cite{flyclient}}
		The verifier queries the prover for the header and MMR proof for a single block $k$ in the prover's chain of $n+1$ blocks.
		\begin{center}
			\textbf{Verifier}
		\end{center}
		\begin{enumerate}
			\item Has the root of the MMR of $n$ blocks stored in the $n+1$ block's header 
			\item Queries prover for the header of block $k$ and for $\Pi_{k \in n}$
			\item Verifies that the hashes of $\Pi_{k \in n}$ hash up to the root of MMR$_n$
			\item Calculates the root of the MMR of the $k-1$ blocks from $\Pi_{k \in n}$ 
			\item Compares the calculated root with the root in the header of block $k$
			\item If everything checks out, accepts the block proof
		\end{enumerate}
		\begin{center}
			\textbf{Prover}
		\end{center}
		\begin{enumerate}
			\item Has chain of $n+1$ blocks and the MMR of the first $n$ blocks 
			\item Receives query for block $k$ from verifier 
			\item Calculates $\Pi_{k \in n}$ from MMR$_n$
			\item Sends header of $k$ and $\Pi_{k \in n}$ to verifier
		\end{enumerate}
	\end{algorithm}

	For a single block query the protocol can be described as follows. The verifier is synchronized to a chain $\chain$ and already has the tip of the chain $\chain[-1]$. He then queries the prover and receives the header of the specific block of interest $B$ in $\chain$ and the inclusion proof $\Pi_{B \in \chain}$. Then the verifier checks the validity of $B$ in the same way as already described for the random sampled blocks in the synchronization protocol. 
	The prover/verifier single query protocol is given in Algorithm~\ref{alg:flyclient_iinfix_protocol}.


\end{section}

\begin{section}{Velvet MMRs}
	A velvet fork suggests that any protocol changes are deployed in a backwards-compatible manner so that unupgraded players accept upgraded blocks and upgraded players accept unupgraded blocks too. In practice, the protocol changes are applied via some auxiliary data included in each block, which make sense and are used only by upgraded parties,while being omitted as comments by unupgraded parties. 

	In the context of FlyClient, velvet fork deployment implies that upgraded miners additionally include an MMR root in each block's header. The claim made in the paper is that considering a constant fraction $\alpha$ of upgraded blocks in the chain, an honest prover could produce proofs by utilizing only these blocks and by joining the intermediary blocks together. This should result to less efficient proofs, bacause in order to random sample a sufficient number of upgraded blocks you need a larger underlying chain than in a hard or soft fork, since only a portion of the blocks are upgraded. The claim is that the velvet proofs remain secure. We show that this claim does no hold by presenting a specific attack.
	
	Velvet fork requires any block to be accepted in the chain regardless the validity of the auxiliary data coming with the protocol update. In the case of FlyClient, an adversary may produce blocks which are compatible to the basic consensus rules but contain invalid MMR information. As an example, an invalid MMR may omit blocks existing in $\chain$ or contain blocks which belong in temporary forks of $\chain$.  We call adversarially generated blocks containing invalid MMRs \emph{thorny} blocks. A specific case of thorny is illustrated in Figure~\ref{fig:thorny_flyclient}.

	\begin{figure*}[h!]
		\begin{center}
			\includegraphics[width=0.5\textwidth]{figures/thorny_flyclient.pdf}
		\end{center}
		\caption{A thorny block colored black containing invalid MMR commitment to a block of a fork chain illustrated as a dashed arrow. With respect to the MMR commitments the black block along with the grey ones form a chain.}
		\label{fig:thorny_flyclient}
	\end{figure*}
	
\end{section}

\begin{section}{The Attack}
	The  velvet FlyClient description does not deal with thorny blocks, meaning blocks that contain only seemingly valid auxiliary data. More specifically, it remains unspecified whether blocks containing an MMR root but not the correct one are considered valid upgraded blocks or unpupgraded blocks. We work on the hypothesis that honest miners validate the MMR root of the blocks and blocks containing invalid MMR roots are treated as unupgraded. This seems to be the only reasonable option. In the opposite case any block containing trash data in the place where the MMR root should be would completely destroy the protocol making it impossible to deliver a valid proof. We will now describe the chainsewing attack against the velvet FlyClient protocol.

	Consider that the adversary utilizes more than one thorny blocks in order to cut-and-paste portions from the chain adopted by honest parties to his fork chain. 
	Consider the attack illustrated in Figure~\ref{fig:simple_chainsewing_flyclient}. The adversary acts as follows. She first mines upgraded blocks on a fork chain $\chain_A$ until she generates block $b'$ containing a double spending attack. Afterwards she mines block $a'$ in the honest chain $\chain_B$, which includes an MMR root for her fork chain, thus including the blocks from genesis and up to $b'$. After that she keeps mining blocks on $\chain_B$, which contain MMR root that builds on top of the root included in $a'$, including only the following adversarially generated blocks in $\chain_B$ and ignoring any intermediary honestly generated blocks while constructing the MMRs of her blocks. Additionally, during this period when she mines on $\chain_B$ she tries to suppress any honest upgraded block in $\chain_B$. Towards this end she acts as follows. She regularly mines block on $\chain_B$ as described ignoring honestly unupgraded blocks. When an honest upgraded block $\chain[i]$ is appended she mines on top of block $\chain[i-1]$. If she mines a block and the suppression fails she can still use her fresh block in her proof by continuing to construct consistent MMRs in the following blocks as described before. Figure~\ref{fig:combined_chainsewing_flyclient} illustrates an example of the underlying suppression attack.
	From the verifier's perspective the ignored honest blocks are simply perceived as unupgraded blocks. At some later point, the adversary generates block $a$ in the fork chain, which also contains an MMR root for all the grey and black blocks up to block $b$. Right afterwards the adversary produces a proof as described in the velvet FlyClient protocol giving block $a$ as $\chain[1]$ and the count number of all the grey and black blocks. From the verifier's perspective the black and grey colored blocks form a valid chain, since the head of the chain $a$ contains consistent MMR commitments with all these blocks. In addition, the random sampling performed by the FlyClient protocol will succeed because there are no invalid blocks in this chain.

	\begin{figure*}[h!]
		\begin{center}
			\includegraphics[width=0.85\textwidth]{figures/simple_chainsewing_flyclient.pdf}
		\end{center}
		\caption{Chainsewing attack. Two thorny blocks $a, a'$ are used to chainsew a portion of honest chain $\chain_B$ to adversarial fork $\chain_A$. Black blocks imply adversarially generated blocks. Grey blocks are used in the adversarial proof along with the black ones. Wavy lines imply one or more blocks. Dashed arrows imply an MMR commitment for the destination block in the block of origin.}
		\label{fig:simple_chainsewing_flyclient}
	\end{figure*}

	\begin{figure*}[h!]
		\begin{center}
			\includegraphics[width=0.85\textwidth]{figures/flyclient_attack_suppression.pdf}
		\end{center}
		\caption{Chainsewing along with suppression attack. Black blocks imply adversarially generated blocks. Blue blocks imply honest upgraded blocks,which the adversary tries to suppress. Wavy lines imply one ore more blocks. Dashed arrows imply an MMR commitmens.}
		\label{fig:combined_chainsewing_flyclient}
	\end{figure*}

\end{section}