\section{The Backbone Model}
\subsection{The protocol}
The Backbone protocol is executed by an arbitrary number of parties over an unauthenticated network.
We consider $n$ parties in total, $t$ of which may be controlled by an adversary.

Table \ref{table:backbone_parameters} contains all the parameters of the Backbone protocol and will
be a point of reference throughout this work.

\begin{table}[h!]
	\caption{\textbf{ \footnotesize The parameters of backbone model analysis. Positive integers $n, t, L, s, l, T, k,
	 \kappa$, positive reals $f, \epsilon, \delta, \mu_Q, \tau, \lambda$ where
	 $f, \epsilon, \delta, \mu_Q \in (0,1)$.}}
	\begin{tabular}{| p{14.74cm} |}
		\hline
		$\lambda : \text{security parameter}$\\
		$\kappa : \text{length of the hash function output}$\\
		$n : \text{number of parties mining, } t \text{ of which are controlled by the adversary}$\\
		$T : \text{the target hash value used by parties for solving POW}$\\
		$t :  \text{number of parties controlled by the adversary}$\\
		$\delta : \text{advantage of honest parties}, \dfrac{t}{n-t} \leq 1-\delta$\\
		$f : \text{probability at least one honest party succeeds in finding a POW in a round}$\\
		$\epsilon : \text{random variables' quality of concentration in typical executions}$\\
		$k : \text{number of (suffix) blocks for the common prefix property}$\\
		% be more specific in l please\\
		$l : \text{number of blocks for the chain quality property}$\\
		$\mu_Q : \text{chain quality parameter}$\\
		%% more specific for s too please
		$s : \text{number of rounds for the chain growth property}$\\
		$\tau : \text{chain growth parameter}$\\
		$L : \text{the total run-tiem of the system}$\\
		\hline
	\end{tabular}
	\label{table:backbone_parameters}
\end{table}

We will now give a high-level description of the Backbone Protocol and its fundamental components,
namely the three supporting algorithms for \textit{chain validation}, \textit{chain comparison} and
\textit{proof of work}. We will also define and discuss the three properties of the protocol, namely
\textit{Common Prefix}, \textit{Chain Quality} and \textit{Chain Growth}. For a more formal and
detailed presentation refer to the Backbone paper\cite{backbone}.

Consider that the protocol has already run for some rounds and a chain $C$ has been formed. Consider
also an honest party that wishes to connect to the network, obtain the up-to-date version of the
chain and try to extend it.
The honest party connects to the network and first tries to synchronize to the current chain. The
chain synchronization takes two steps to conclude. First, the newly connected peer receives a number
of candidate chains by other peers in the network and validates them one by one as for the structural 
properties of each block \textit{(Chain Validation)}. In particular, for each block the chain
validation algorithm checks that the proof-of-work is properly solved, that the hash of the previous
block is properly included in the block and that the the rest of the information included satisfies
a certain validity predicate $V(\cdot)$ depending on the application. For example, in Bitcoin
application it is checked that all the included transactions are valid according to the
UTXO set.

Afterwards, the \textit{Chain Comparison} algorithm is applied, where all the valid chains are
compared to each other and the longest one, as for total number of blocks or total hashing power
included, is considered the current active chain.

At last, in order to expand the chain by appending one more block to it, the \textit{Proof Of Work}
algorithm is applied, where the miner attempts to solve a proof of work as follows. The miner
constructs the contents of the block, including the hash of the previous block and a number of new
transactions published to the network. Consider that he can calculate the value $h = G(s,x)$ up
to this point. Finally it remains to compute the \textit{ctr} value so that $H(ctr, h) < T$. The
protocol is running in rounds and each party can make at most $q$ queries to function $H(\cdot)$
within a single round. If a suitable \textit{ctr} is found, an honest party quits any queries
remaining and announces the new born block to the network.

\subsection{Basic properties}
We can now define the three desired properties of the backbone protocol.

\begin{definition}[Common Prefix Property]
	The common prefix property $Q_{cp}$ with parameter
$k \in \mathbb{N}$ states that for any pair of honest players $P_1, P_2$ adopting the chains
$C_1, C_2$ at rounds $r_1 \leq r_2$ respectively, it holds that $C_1^{\lceil k} \preceq C_2$.
	\label{defn:common_prefix}
\end{definition}


\begin{definition}[Chain Quality Property]
	The chain quality property $Q_{cq}$ with parameters $\mu_{cq} \in \mathbb{R}$ and $l \in \mathbb{N}$ states that for any honest party $P$ with chain $\mathcal{C}$ it holds that for any $l $ consecutive blocks of $\mathcal{C}$ the ratio of honest blocks is at least $\mu_{cq}$.
	\label{defn:chain_quality}
\end{definition}

\begin{definition}[Chain Growth Property]
	The chain growth property $Q_{cg}$ with
parameters $\tau \in \mathbb{R}$ and $s \in \mathbb{N}$ states that for any honest party
\textit{P} with chain $\mathcal{C}$, it holds that for any $s$ rounds there are at least 
$\tau \cdot s$ blocks added to the chain of \textit{P}.
	\label{defn:chain_growth}
\end{definition}