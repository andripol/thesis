\documentclass[11pt,a4paper]{article}
\usepackage[utf8]{inputenc}
\usepackage{amsmath}
\usepackage{amsfonts}
\usepackage{amssymb}

\usepackage{import}

\usepackage{graphicx}
\usepackage[colorlinks=true,citecolor=magenta,linkcolor=blue]{hyperref}

\begin{document}

\begin{center}
\begin{Large}
\textbf{NIPoPoWs under Velvet Fork\\}
\end{Large}

\end{center}

\section{Introduction}
\import{sections/introduction}{intro.tex}

\section{Model Definition and Notation}

\section{NIPoPoWs under Soft or Hard Fork}
\import{sections/nipopows_hard_fork/}{nipopows_hard_fork.tex}

\section{NIPoPoWs under Velvet Fork}
\import{sections/nipopows_velvet_fork/}{nipopows_velvet_fork.tex}


\subsubsection{Security of Suffix Proofs}
Our security proof is based on the security proof of NIPoPoWs Suffix Security Proof under a soft or hard fork. Consequently, most of the Definitions and Lemmas presented in Section \ref{proof_under_hard_fork} will be referenced and applied for the security proof under velvet fork conditions. Keep in mind that we operate under the condition of (1/3)-bounded adversary.

The notion of Chain Quality is decisive for the operation of NIPoPoWs under a velvet fork. Before stepping into the security proof we provide the formal definition of this notion and the basic Theorem where its basic metric is computed.\\

\textbf{Definition 3 }(Chain Quality Property)\cite{Backbone}.The chain quality property $Q_{cq}$ with parameters $\mu \in \mathbb{R}$ and $l \in \mathbb{N}$ states that for any honest party $P$ with chain $C$ it holds that for any $l $ consecutive blocks of $C$ the ration of honest blocks is at least $\mu$.\\

\textbf{Theorem 3 }(Chain Quality)\cite{Backbone}. \textit{In a typical execution the chain quality property holds with parameter $\mu > 1 - (1 + \frac{\delta}{2}) \cdot \frac{t}{n-t} - \frac{\delta}{2}$.}\\
\textit{Proof.} See \cite{Backbone}.\\

%As a consequence from the protocol patch suggested, adversarially generated blocks with incorrect interlinks can only be included in the adversary's suffix proofs. For an honest player constructing a suffix proof these blocks are as if not existing in his chain, whereas the adversary may use the most of them in her suffix proof in the worst case. It follows that from the perspective of suffix proofs these blocks can be conceived as participating in the adversary's chain since this would be the perception of the verifier. The adversary may or may not use the most of these blocks, so considering them in advance as part of the adversary's chain could only make our proof stronger for the main case. 

\textbf{Theorem 4. (Security under velvet fork)} \textit{Assuming honest majority under velvet fork conditions such that $\dfrac{t}{n-t} \leq \dfrac{1-\delta}{2}$, the non-interactive proofs-of-proof-of-work construction for computable $\kappa$-stable monotonic suffix-sensitive predicates under velvet fork conditions is secure with overwhelming probability in $\kappa$.}\\

\textit{Proof.} By contradiction. We follow the proof construction of Theorem 2 and extend it. Let $Q$ be a $\kappa-$stable monotonic suffix-sensitive chain predicate. Assume NIPoPoWs under velvet fork on $Q$ is insecure. Then, during an execution at some round  $r_3$, $Q(C)$ is defined and the verifier $V$ disagrees with some honest participant. Assume the execution is typical. $V$ communicates with adversary $A$ and honest prover $B$. The verifier receives proofs $\pi_A, \pi_B$. Because $B$ is honest, $\pi_B$ is a proof constructed based on underlying blockchain $C_B$ (with $\pi_B \subseteq C_B$), which $B$ has adopted during round $r_3$ at which $\pi_B$ was generated. Consider $C_A$ the chain containing at least some of the blocks in $\pi_A$, while the remaining blocks $\pi_A$ must belong in $C_B$. 
The verifier outputs $\neg Q(C_B)$. Thus it is necessary that $\pi_A \geq \pi_B$. We show that $\pi_A \geq \pi_B$ is a negligible event. 
Let $b = LCA(\pi_A, \pi_B)$. Let the levels of comparison decided by the verifier be $\mu_A$ and $\mu_B$ respectively. Let $\mu'_B$ be the adequate level of proof $\pi_B$  with respect to block $b$. Call $\alpha_A = \pi_A \uparrow^{\mu_A}\{b:\}$, 
$\alpha'_B = \pi_B \uparrow^{\mu'_B}\{b:\}$.

%%%%%%%%   reconsider this paragraph
Our proof construction is based on the following scheme: we show that the competing suffix proofs can be conceived as consisting of three distinct parts. Each part denotes a specific round set and is called after the number of blocks existing in $\pi_A$ for that round set. Part $k_1$ lies for the first part of the proofs between blocks $b = LCA(\pi_A, \pi_B)$ and $b_2 = LCA(C_A, C_B)$ meaning for the common 0-level part of $\alpha_A,  \alpha_B$. Part $k_2$ lies for the second part of the proofs considering the rounds from block $b_2 = LCA(C_A, C_B)$ up until the Common Prefix is established at the 0-level chains for that fork point. The third and last part, $k_3$ lies for the rest blocks in the proofs.\\
The above are illustrated, among other, in Parts I, II of Figure \ref{fig:proof_velvet}.

\begin{figure}[h!]
	\begin{center}
		\includegraphics[scale=0.5]{figures/proof_velvet.png}
	\end{center}
	\caption{\textit{ Wavy lines imply one or more blocks. Dashed lines and arrows imply interlink pointers to superblocks. \textbf{I}: the three round sets in two competing proofs at different levels, \textbf{II}: the corresponding 0-level chains, \textbf{III}: blocks participating in chains $C_B$, $C_A$ as conceived by the verifier's perspective.}}
	\label{fig:proof_velvet}
\end{figure}

We will now show three successive claims under velvet fork conditions: First, $\alpha_A \downarrow \uparrow^{\mu_A}$ and $\alpha'_B \downarrow$ are mostly disjoint. Second, $a_A$ contains mostly adversarially generated blocks. And third, the adversary is able to produce this $a_A$ with negligible probability.\\
Let $\alpha_A = k_1 + k_2 + k_3$ and let $k_1, k_2, k_3$ be as defined in the following Claims.\\
Let round $r_1$ be the round when block $b$ is generated and round $r_2$ when block $b_2 = LCA(\alpha_A, \alpha'_B\downarrow)$ is generated.\\

\textbf{Claim 1:} As for honestly generated blocks, $\alpha_A$ and $ \alpha'_B\downarrow$ are mostly disjoint. Following the proof of Theorem 2 we conclude that $\vert \alpha_A\downarrow\uparrow^{\mu_A}[1:] \cap \alpha'_B\downarrow[1:] \vert \leq k_{1} = 2^{\mu'_B - \mu_A}$. In order to see this under the velvet fork conditions consider that the adversary behaves honestly for blocks in her proof between $b$ and $b_2$, where Claim1 of Theorem 2 applies directly. In the opposite case, the adversary includes a block with false interlink after block $b$ and before block $b_2$ and because of Lemma 5 no more honestly generated blocks can be included in $\alpha_A$ and we can immediately proceed to Claim 3 of this proof.

So we conclude that there are at least $\vert \alpha_A \vert - k_1$ blocks after block $b$ in $\alpha_A$ which are not honestly generated blocks existing in $\alpha'_B\downarrow$. In other words, there are $\vert \alpha_A \vert - k_1$ blocks after block $b$ in $\alpha_A$, which are either adversarially generated existing in $\alpha_B\downarrow$ either don't belong in $\alpha_B\downarrow$. This makes $b_2$ the last block before the 0-level fork point included in the adversary's proof.\\

\textbf{Claim 2.} 
At least $k_3$ superblocks of $\alpha_A$ are adversarially generated. Just as the proof of Theorem 2 and using a similar notation, because of the Common Prefix property on parameter $k_{2\downarrow}$, $\alpha_A[k_{1}+k_{2}:]$ could contain no honestly generated blocks. In order to see this for the velvet fork conditions let's again consider the case that the adversary behaves honestly for the first $(k_1 + k_2)$ blocks of her proof in which case Claim 2 of Theorem 2 is immediately applied. In the opposite case, consider that the adversary includes in her proof a block with invalid interlink at some earlier point. Again, because of Lemma 5 no more honestly generated blocks can be included in $\alpha_A$ and we can proceed to Claim 3 of this proof.

%\begin{figure}[h!]
%	\begin{center}
%		\includegraphics[scale=0.5]{figures/exclude.png}
%	\end{center}
%	\caption{\textit{ Wavy lines imply one or more blocks. Dashed arrows imply interlink pointers to superblocks. Adversarially generated blocks are colored black. Grey colored blocks may be honestly or adversarially generated. \textbf{I}: the 0-level chains, \textbf{II}: the corresponding proof chains; some blocks generated in $C_A$ are excluded from proof $\pi_A$ in favor of the sewed blocks from $C_B$.}}
%	\label{fig:exclude}
%\end{figure}

From all the above Claims we have that:\\
In the first round set, because of the common underlying chain:
\begin{equation*}
2^{\mu_A} \vert \alpha_A^{k_{1a}} \vert \leq 2^{\mu'_B} \vert \alpha'{_B^{k_{1a}}} \vert
\end{equation*}
and because of Lemma 1 we obtain with high probability:
\begin{equation*} 
2^{\mu_A} \vert \alpha_A^{k_{1b}} \vert \leq 2^{\mu'_B} \vert \alpha'{_B^{k_{1b}}} \vert
\end{equation*}
So finally:
\begin{equation} \label{eq_v_round_set_1}
2^{\mu_A} \vert \alpha_A^{k_1} \vert \leq 2^{\mu'_B} \vert \alpha'{_B^{k_1}} \vert
\end{equation}
In the second round set:\\
(To be completed.)\\
In the third round set, considering the equivalent problem after reforming the chains because of Honest Majority Assumption under Velvet Fork and Lemma 1 we have:
\begin{equation} \label{eq_v_round_set_3}
2^{\mu_A} \vert \alpha_A^{k_3} \vert < 2^{\mu'_B} \vert \alpha'{_B^{k_3}} \vert
\end{equation}

\subsection{Infix Proofs}

\bibliographystyle{plain} % We choose the "plain" reference style
\bibliography{refs} % Entries are in the "refs.bib" file

\end{document}