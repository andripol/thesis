\documentclass[11pt,a4paper]{article}
\usepackage[utf8]{inputenc}
\usepackage{amsmath}
\usepackage{amsfonts}
\usepackage{amssymb}

\usepackage{import}

\usepackage{graphicx}
\usepackage[colorlinks=true,citecolor=magenta,linkcolor=blue]{hyperref}

\begin{document}

\begin{center}
\begin{Large}
\textbf{NIPoPoWs under Velvet Fork\\}
\end{Large}

\end{center}

\section{Introduction}
\import{sections/introduction}{intro.tex}

\section{Model Definition and Notation}

\section{NIPoPoWs under Soft or Hard Fork}
\import{sections/nipopows_hard_fork/}{nipopows_hard_fork.tex}

\section{NIPoPoWs under Velvet Fork}
\import{sections/nipopows_velvet_fork/}{nipopows_velvet_fork.tex}


\subsubsection{Security of Suffix Proofs}
Our security proof is based on the security proof of NIPoPoWs Suffix Security Proof under a soft or hard fork. Consequently, most of the Definitions and Lemmas presented in Section \ref{proof_under_hard_fork} will be referenced and applied for the security proof under velvet fork conditions. Keep in mind that we operate under the condition of (1/3)-bounded adversary.

\textbf{Theorem 3. (Security under velvet fork)} \textit{Assuming honest majority under velvet fork conditions, the non-interactive proofs-of-proof-of-work construction for computable $\kappa$-stable monotonic suffix-sensitive predicates under velvet fork conditions is secure with overwhelming probability in $\kappa$.}\\

\textit{Proof.} By contradiction. We follow the construction of Theorem 2 proof and extend it. Let $Q$ be a $\kappa-$stable monotonic suffix-sensitive chain predicate. Assume NIPoPoWs under velvet fork on $Q$ is insecure. Then, during an execution at some round  $r_3$, $Q(C)$ is defined and the verifier $V$ disagrees with some honest participant. Assume the execution is typical. $V$ communicates with adversary $A$ and honest prover $B$. The verifier receives proofs $\pi_A, \pi_B$. Because $B$ is honest, $\pi_B$ is a proof constructed based on underlying blockchain $C_B$ (with $\pi_B \subseteq C_B$), which $B$ has adopted during round $r_3$ at which $\pi_B$ was generated. Consider $C_A$ the chain containing at least some of the blocks in $\pi_A$, while the remaining blocks $\pi_A$ must belong in the $C_B$. 
The verifier outputs $\neg Q(C_B)$. Thus it is necessary that $\pi_A \geq \pi_B$. We show that $\pi_A \geq \pi_B$ is a negligible event. 
Let $b = LCA(\pi_A, \pi_B)$. Let the levels of comparison decided by the verifier be $\mu_A$ and $\mu_B$ respectively. Let $\mu'_B$ be the adequate level of proof $\pi_B$  with respect to block $b$. Call $\alpha_A = \pi_A \uparrow^{\mu_A}\{b:\}$, 
$\alpha'_B = \pi_B \uparrow^{\mu'_B}\{b:\}$.\\
The above are illustrated, among other, in Parts I, II of Figure \ref{fig:proof_velvet}.
\begin{figure}[h!]
	\begin{center}
		\includegraphics[scale=0.5]{figures/proof_velvet.png}
	\end{center}
	\caption{\textit{ Wavy lines imply one or more blocks. Dashed lines and arrows imply interlink pointers to superblocks. \textbf{I}: the three round sets in two competing proofs at different levels, \textbf{II}: the corresponding 0-level chains, \textbf{III}: the reformed chains $C_B$, $C_A$ so that adversarially generated blocks with false interlink participating in cyclic dependency with $C_A$ are removed from $C_B$ and added to $C_A$.}}
	\label{fig:proof_velvet}
\end{figure}
We will now show three successive claims under velvet fork conditions: First, $\alpha_A \downarrow \uparrow^{\mu_A}$ and $\alpha'_B \downarrow$ are mostly disjoint. Second, $a_A$ contains mostly adversarially generated blocks. And third, the adversary is able to produce this $a_A$ with negligible probability. Note that the notation $\alpha_A \downarrow \uparrow^{\mu_A}$ denotes the blocks of $\alpha_A$ which belong in the 0-level chain $C_A$, thus are not sewed in the adversary's proof by $C_B$.\\
Let $\alpha_A = k_1 + k_2 + k_3$ and let $k_1, k_2, k_3$ be as defined in the following Claims.\\
Let round $r_1$ be the round when block $b$ is generated and round $r_2$ when block $b_2 = LCA(\alpha_A \downarrow \uparrow^{\mu_A}, \alpha'_B\downarrow)$ is generated.\\
\textbf{Claim 1:} $\alpha_A \downarrow \uparrow^{\mu_A}, \alpha'_B\downarrow$ are mostly disjoint. Following the proof of Theorem 2 we conclude that $\vert \alpha_A\downarrow\uparrow^{\mu_A}[1:] \cap \alpha'_B\downarrow[1:] \vert \leq k_{1} = 2^{\mu'_B - \mu_A}$. In order to see this under the velvet fork conditions consider the following two claims. First let block $b_2 = LCA(\alpha_A \downarrow \uparrow^{\mu_A}, \alpha'_B\downarrow)$. This means that $b_2$ is the LCA block considering the adversary's proof and honest player's 0-level chain excluding any adversarially generated block included in $\alpha_A$ after the 0-level fork point.\\
\textit{\underline{Claim 1a}:} If no adversarially generated block of chain $C_B$ with incorrect interlink is included in $\alpha_A$ until block $b_2$ then the claim's proof is the same to that of Theorem 2, thus we have $k_1 \leq 2^{\mu'_B - \mu_A}$.\\
\textit{\underline{Claim 1b}:}  If a block $b' = \{b' \in C_B\{b:b_2\} \cap \textit{ } b' \textit{ contains incorrect interlink} \}$ is included in $\alpha_A$ then we have that $\vert \alpha_A\downarrow\uparrow^{\mu_A}[1:] \cap \alpha'_B\downarrow[1:] \vert \leq k_1 = k_{1a} + k_{1b}$, where $ k_{1a} = 2^{\mu'_B - \mu_A}$. 
	
Note that after block $b'$ no more honestly generated blocks of the common subchain $C_B\{b':b_2\}$ are included in $\alpha_A$ because of Remark 2. Blocks in $\alpha_A\{b':b_2\}$ are all adversarially generated of $C_B$. Thus from that point on it is equivalent considering two disjoint chains, one with honestly and another with adversarially generated blocks, competing each other. Let these adversarially generated blocks count to $k_{1b}$,  where $k_{1b} \leq C_B\{b':b_2\}\uparrow^{\mu_A}$.\\ \textit{(In the end, consider the rounds when these blocks were generated and apply Lemma 1 to get a desirable inequality)}\\
From all the above, we conclude that there are $\vert \alpha_A \vert - k_1$ blocks after block $b$ in $\alpha_A$ which are not honestly generated blocks existing in $\alpha_B\downarrow$. In other words, there are $\vert \alpha_A \vert - k_1$ blocks after block $b$ in $\alpha_A$, which are either adversarially generated existing in $\alpha_B\downarrow$ either don't belong in $\alpha_B\downarrow$. This makes $b_2$ the last block before the 0-level fork point included in the adversary's proof.\\
\textbf{Claim 2.} 
At least $k_3$ superblocks of $\alpha_A$ are adversarially generated. Just as the proof of Theorem 2 and using a similar notation, because of the Common Prefix property on parameter $k_{2a\downarrow}$, $\alpha_A \downarrow \uparrow [k_{1a}+k_{2a}:]$ could contain no honestly generated blocks. In case there are no sewed blocks, the above holds for $\alpha_A$ too. In order to generalize to velvet fork conditions, let the sewed blocks count to $k_{2b}$ and the sum $k_2 = k_{2a} + k_{2b}$. We conclude that $\alpha_A [k_{1}+k_{2}:]$ could contain no honestly generated blocks. \\ 
Note that for every sewed subchain the adversary includes in her proof, it may be that some other blocks generated in the adversary's chain are left out of the proof. This is due to the way blocks are generated in order to form a valid chain. So, in a case where a number of $C_B$ blocks are sewed in $C_A$ between blocks $b_{s_1}$ and $b_{s_2}$, say at rounds $s_1, s_2$ accordingly, any block generated in $C_A$ in the round set $s_1..s_2$ would be left out of the proof. This is illustrated in Figure \ref{fig:exclude}.\\
\begin{figure}[h!]
	\begin{center}
		\includegraphics[scale=0.5]{figures/exclude.png}
	\end{center}
	\caption{\textit{ Wavy lines imply one or more blocks. Dashed arrows imply interlink pointers to superblocks. Adversarially generated blocks are colored black. Grey colored blocks may be honestly or adversarially generated. \textbf{I}: the 0-level chains, \textbf{II}: the corresponding proof chains; some blocks generated in $C_A$ are excluded from proof $\pi_A$ in favor of the sewed blocks from $C_B$.}}
	\label{fig:exclude}
\end{figure}
%
%Claim * calculates an upper bound of the adversarially generated blocks that $\alpha_A, \alpha'_B\downarrow$ may have in common.\\
%\textit{\underline{Claim *}:} Now consider the blocks containing incorrect interlink. By the time such a block is included in $\alpha_A$, then no other honestly generated block of chain $C_B$ will be included in $\alpha_A$ because of Remark 1 (no honestly generated block included in proof can point to a superblock containing false interlink). Thus, in that case it is equivalent to consider two different chains, the one of the adversary and the other of the honest players, that compete each other. This case may appear after $k_{1a}$ blocks in the $\alpha_A$ as described in the above Claims. Let $b_1 = \{GCA(C_B, \alpha_A) \cap b_1 \textit{ contains false interlink}\}$ and block $b_2 = LCA(C_B, \alpha_A)$. Thus we have 
%$\vert \alpha_A[1:] \cap \alpha'_B\downarrow[1:] \vert = k_{1a} + k_{1b}$, where $k_{1b} \leq C_B\{b_1:b_2\}\uparrow^{\mu_A}$.\\
In the following we consider the reformed chains $C_A, C_B$ for blocks in the $k_3$ region of the proofs. This reforming suggests the following: all adversarially generated blocks with false interlink in subchain $C_B\{b_2:\}$ which participate in cyclic dependency with fork chain $C_A$, are removed from $C_B$ and added in $C_A$ just as the interlink pointers forming the cyclic dependency point out. This reforming is helpful for the purposes of our proof construction and does not change the facts of the problem, or put it in another way, gives us an equivalent problem to solve. This is true, because of Remark 1, that such blocks cannot be included in an honest NIPoPoW proof while can be included in the proof constructed by the adversary, thus from the NIPoPoW suffix proofs perspective these block can be perceived as belonging in the adversary's fork chain. This remark is considered in the following Claim.\\
\textbf{Claim 3.} We reform the chains $C_A, C_B$ as illustrated in Figure 5 (I, II) and Figure \ref{fig:claim3} (II, III). After this reforming we conclude to an equivalent set of compared subchains. Note that the problems before and after the reforming are equivalent because of Remarks 1 and 2. Then this Claim is exactly the same as that of Theorem 2.
In order to complete the proof we work in the same way as in Theorem 2, by considering the three parts of $\alpha_A$ proof ....
From all the above Claims we have that:\\
In the first round set, because of the common underlying chain:
\begin{equation*}
2^{\mu_A} \vert \alpha_A^{k_{1a}} \vert \leq 2^{\mu'_B} \vert \alpha'{_B^{k_{1a}}} \vert
\end{equation*}
and because of Lemma 1 we obtain with high probability:
\begin{equation*} 
2^{\mu_A} \vert \alpha_A^{k_{1b}} \vert \leq 2^{\mu'_B} \vert \alpha'{_B^{k_{1b}}} \vert
\end{equation*}
So finally:
\begin{equation} \label{eq_v_round_set_1}
2^{\mu_A} \vert \alpha_A^{k_1} \vert \leq 2^{\mu'_B} \vert \alpha'{_B^{k_1}} \vert
\end{equation}
In the second round set:\\
(To be completed.)\\
In the third round set, considering the equivalent problem after reforming the chains because of Honest Majority Assumption under Velvet Fork and Lemma 1 we have:
\begin{equation} \label{eq_v_round_set_3}
2^{\mu_A} \vert \alpha_A^{k_3} \vert < 2^{\mu'_B} \vert \alpha'{_B^{k_3}} \vert
\end{equation}

\subsection{Infix Proofs}

\bibliographystyle{plain} % We choose the "plain" reference style
\bibliography{refs} % Entries are in the "refs.bib" file

\end{document}