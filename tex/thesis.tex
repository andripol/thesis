\documentclass[11pt,a4paper]{article}
\usepackage[utf8]{inputenc}
\usepackage{amsmath}
\usepackage{amsfonts}
\usepackage{amssymb}

\usepackage{import}

\usepackage{graphicx}
\usepackage[colorlinks=true,citecolor=magenta,linkcolor=blue]{hyperref}

\usepackage[ruled,vlined]{algorithm2e}

\begin{document}

\begin{center}
\begin{Large}
\textbf{NIPoPoWs under Velvet Fork\\}
\end{Large}

\end{center}

\section{Introduction}
\import{sections/introduction}{intro.tex}

\section{Model Definition and Notation}
Our analysis concerns proof-of-work cryptocurrencies and is based on the standard Backbone 
model\cite{Backbone}.

\subsection{Blockchain Preliminaries}
A blockchain, or simply chain, is a timely ordered sequence of blocks.  In a cryptocurrecny
blockchain, like Bitcoin, a block is a proof-of-work verified set of information about a number of
transactions, the previous block in the block sequence and a nonce. The proof-of-work involves a
computation over a cryptographic puzzle. More specifically, it involves scanning for a value called
nonce, that when included in the block the total hash of the block results to a value lower than a
certain threshold.

More formally, let $G(\cdot)$, $H(\cdot)$ be cryptographic hash functions. A \textit{block} is a
triple of the form $B = \langle s, x, ctr \rangle$, where $s$ is the previous block \textit{id}, $x$
is the transactions information and $ctr \in \mathbb{N}$, such that satisfy the predicate
$validBlock^T(B)$ defined as
\begin{center}
\begin{equation}
	H(ctr, G(s,x)) < T
\end{equation}
\end{center}

The threshold parameter $T \in \mathbb{N}$ is called the block's \textit{difficulty level}.
Throughout this work we consider a constant value for the threshold \textit{T}, although this is not
the case in a real proof-of-work blockchain.

The rightmost block is the \textit{head} the chain and is called the \textit{Genesis} block often
denoted \textit{G}, while the whole chain is denoted \textit{C}. So a chain \textit{C} with
$G = \langle s, x, ctr \rangle$ can be extended by appending a block $B = \langle s', x', ctr'
\rangle$ as long as it holds that $s' = H(ctr, G(s,x))$. In effect every block is connected to the
previous block in the chain by containing its hash. This is called the \textit{prevId} relationship.
Figure \ref{fig:abstract_chain} provides a high level representation of a blockchain including the
bootstrap step of the very first block in the chain, where instead of the \textit{prevId},
arbitrary data may be included in \textit{s}.

\begin{figure}[h!]
	\begin{center}
		\includegraphics[scale=0.49]{figures/abstract_chain_bootstrap.png}
	\end{center}
	\caption{\textit{A high-level representation of a blockchain. }}
	\label{fig:abstract_chain}
\end{figure}


Consider a peer-to-peer network where each party may have one of the following three roles:
lightweight \textit{clients}, full \textit{nodes} and \textit{miners}.
Miners maintain an updated copy of the chain locally, while providing computational power, also
called hashpower, to extend it. In order to extend the chain by one block, the miner has to perform
a proof-of-work as already described.
Full nodes can be thought of as miners with zero hashpower. Full nodes are also called
\textit{provers}, since they provide proofs  answering the queries for specific chain information
made by lightweight clients, according to Simplified Payment Verification (SPV) described by
Nakamoto\cite{Nakamoto}.

According to SPV scheme lightweight clients only need to store the block headers of the longest
valid chain. A block header includes only a Merkle Tree Root of the Merkle Tree comprised by
the transactions included in that specific block. In order to validate that a transaction is
finalized, a client needs to query the nodes until he is convinced that he has the longest
valid chain, search for the block containing that transaction and finally verify an inclusion
proof of the transaction in the block of interest.

\begin{figure}[h!]
	\begin{center}
		\includegraphics[scale=0.25]{figures/SPV_nakamoto.png}
	\end{center}
	\caption{\textit{ High level representation of blockchain data kept by a lightweight client
	 and an inclusion proof for a transaction Tx3.\cite{Nakamoto}  }}
	\label{fig:SPV_nakamoto}
\end{figure}

In the SPV scheme a client needs to store blockchain data of linear size to the whole chain. By
the time of writing Bitcoin's blockchain counts to almost 264GB and is estimated to grow more
than 50GB per year.  Since the growth rate of the chain is rather linear and constant, we need
to construct more efficient protocols serving the needs of lightweight clients. To this end,
the interaction between lightweight clients and full nodes is in our case supported  by the
NIPoPoWs\cite{NIPoPoWs} primitive which allows polylogarithmic poofs to the size of the chain.

\section{NIPoPoWs under Soft or Hard Fork}
\import{sections/nipopows_hard_fork/}{nipopows_hard_fork.tex}

\section{NIPoPoWs under Velvet Fork}
\import{sections/nipopows_velvet_fork/}{nipopows_velvet_fork.tex}


\subsubsection{Security of Suffix Proofs}
\import{sections/nipopows_velvet_fork/}{suffix_security.tex}


\subsection{Infix Proofs}
The security of the original NIPoPoWs protocol suffers under velvet fork conditions for the case of infix proofs as well. Again, since blocks containing incorrect interlink pointers are accepted in the chain, the adversary may create an infix proof for a transaction included in a block mined on a different chain. This attack is presented in detail in the following.

An infix proof attack when applying the original protocol under a velvet fork should be obvious after our previous discussion. So consider the updated protocol for secure suffix proofs as described in the previous section. A problem here is that in the updated protocol some blocks are excluded from the interlink, while we should still be able to provide proofs for transactions included in any block of the chain. 

For this reason, let us initially consider an additional protocol patch suggesting to include a second interlink data structure in each block, which will be updated without any block exclusion, just as described in the original protocol and will be used for constructing infix proofs only. In order to be secure we could think of allowing using pointers of the second interlink only for the \textit{followDown} part of the algorithm. But still, the adversary may use an invalid pointer of a block visited during the \textit{followDown} procedure and jump to a block of another chain providing a transaction inclusion proof concerning that block. This attack is illustrated in Figure \ref{fig:infix_attack}.

\begin{figure}[h!]
	\begin{center}
		\includegraphics[scale=0.52]{figures/infix_attack.png}
	\end{center}
	\caption{\textit{Adversarial fork chain $C_A$ and an adversarial infix proof based on the chain adopted by an honest player. Wavy lines imply one or more blocks. Blocks generated by the adversary are colored black. Dashed arrows represent interlink pointers included in the proof as part of the \textit{followDown} procedure. The adversary provides infix proof for a transaction in block b'. }}
	\label{fig:infix_attack}
\end{figure}

Thus giving the ability to utilize invalid pointers even in a narrow block window can break the security of our protocol. 

\subsubsection*{Protocol patch for NIPoPoWs infix proofs under velvet fork}
However, since we have proved the security of suffix proofs we can include some more information in the blocks participating in these proofs in order to provide secure infix proofs as well.
 
Specifically, we suggest full nodes to maintain an authenticated data structure, let's say a Merkle Tree, for the blocks consisting the longest valid chain at each time point, and each block to additionally contain the Merkle Tree Root of  the blocks' Merkle Tree. In this way an infix proof will consist of a suffix proof in order to obtain the longest valid chain and a block inclusion proof for the block of interest. 

For example, in order to prove that a specific transaction $tx_1$ took place in a block $b'$, the Prover provides
\begin{itemize}
\item a suffix proof $\pi$
\item a block inclusion proof for $b'$, using the blocks' MTR existing in the tip of the suffix proof chain $\pi[-1]$
\item a transaction inclusion proof for $tx_1$ using the transactions' MTR of block $b'$
\end{itemize}

\subsubsection*{The velvet infix prover}
The construction of an infix proof is described in Algorithm \ref{alg:proveInfixVelvet}. In order to keep the algorithm generic enough for any infix-sensitive predicate, we provide the steps needed until the verification of the block of interest and consider the specific predicate answer as trivial to calculate given the block of interest. The infix prover accepts as input the full chain $C$ and a block of interest $b'$ and returns a proof consisting of the Merkle Tree proof of inclusion for $b'$ and a suffix proof.
\vspace{4mm}

\begin{algorithm}[H]
\SetAlgoNoLine
\DontPrintSemicolon
\SetKwProg{Fn}{function}{:}{\text{end function}}
\Fn{ProveInfixVelvet(C, b')}{
	$(\pi , \chi) \leftarrow Prove_{m,k}(C)$\;
	$tip = \pi [-1]$\;
	$\pi_{b'} \leftarrow \text{ConstrMTInclProof}(tip, b'.id )$\;
 
 
 	\Return ($\pi_b', (\pi, \chi)$)\;
}
 \caption{Velvet Infix Prover}
 \label{alg:proveInfixVelvet}
\end{algorithm}

\vspace{4mm}

\subsubsection*{The velvet infix verifier}
The infix proof verification algorithm is described in Algorithm \ref{alg:verifyInfixVelvet}.
Supposing that the verifier has already concluded to the longest valid chain $C$ after accepting and comparing constesting suffix proofs, the infix verification algorithm only has to confirm the Merkle-Tree inclusion proof for the block of interest $b'$.

\vspace{4mm}

\begin{algorithm}[H]
\SetAlgoNoLine
\DontPrintSemicolon
\SetKwProg{Fn}{function}{:}{\text{end function}}
\Fn{VerifyInfixVelvet($\pi_b', (\pi, \chi)$)}{
	$(\pi , \chi) \leftarrow Prove_{m,k}(C)$\;
	$tip = \pi [-1]$\;
	\Return $\text{VerMTInclProof}(tip.MTR_{blocks}, \pi_{b'}, b'.id )$\;
}
 \caption{Velvet Infix Verifier}
 \label{alg:verifyInfixVelvet}
\end{algorithm}

\vspace{4mm}

\bibliographystyle{plain} % We choose the "plain" reference style
\bibliography{refs} % Entries are in the "refs.bib" file

\end{document}