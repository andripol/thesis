\section{Introduction}
Since the release of Bitcoin about a decade ago, the interest in cryptocurrencies has increased tremendously, while a number of other ``altcoins'' have been constructed in the meantime. Given that cryptocurrencies are starting to be considered a generally accepted means of payment and are used for everyday transactions, the issue of efficiently handling cryptocurrencies by light clients, such as smartphones, has become of great importance.

In this work, we consider the problem of optimizing light clients, or ``SPV clients'' as described in the original Bitcoin paper\cite{nakamoto}. As blockchains are ever growing, the main setback for efficient light client applications is the processing of data amount linear to the size of the blockchain, e.g. for synchronization purposes. 

Our work is based on the construction of Non-Interactive Proofs of Proof of Work\cite{nipopows} that achieves SPV proofs of polylogarithmic portion of the blockchain size. The NIPoPoWs construction suggests a protocol update, that could be possibly implemented by a soft or a hard fork. Given the reluctancy of the  Bitcoin community to proceed to such forks, we consider the case of a velvet fork\cite{nipopows}\cite{velvet}, where it suffices only a portion of the total players to be updated.

Under this scope, our contributions come as follows:
\begin{itemize}
	\item We revise the security proof for superblock NIPoPoWs suffix proof protocol and compute a concrete value for the security parameter $m$
  	\item We illustrate that, contrary to claims of previous works, superlight clients designed to work in a soft fork cannot be readily plugged into a velvet fork and expected to work. We present a novel attack termed the \emph{chain-sewing} attack which thwarts the defenses of previous proposals and allows even a minority adversary to cause catastrophic failures.
	\item We propose the first \emph{backwards-compatible superlight client}. We put forth an interlinking mechanism implementable through a velvet fork. We then construct a superblock NIPoPoW protocol on top of the velvet forked chain and show it allows to build superlight clients for various statements regarding the blockchain state via both ``suffix'' and ``infix'' proofs. 
	\item We prove our construction secure in the synchronous static difficulty model against adversaries bounded to $1/4$ of the mining power of the honest upgraded nodes. As such, our protocol works even if a constant minority of miners adopts it.
\end{itemize}
