We  will now describe an explicit attack against the NIPoPoW suffix proof construction 
under velvet fork. As already argued, since the protocol is implemented under
velvet fork, any adversarial block that is mined in the proper way except
containing false interlink data will be accepted as valid. Taking advantage 
of such thorny blocks in the chain, the adversary maintaining one or more forked 
chains could produce suffix proofs containing blocks of any chain.
The attack is described in detail in the following.

Assume that chain $C_B$ was adopted by an honest player B and chain $C_A$, 
a fork of $C_B$ at some point, maintained by adversary A. Assume that the
adversary wants to produce a suffix  proof in order to attack a light client 
to have him adopt a chain which contains blocks of $C_A$. In order to achieve
this, the adversary needs to include a greater amount of total proof-of-work
in her suffix proof, $\pi_A$, in comparison to that included in the honest
player's proof, $\pi_B$, so as to achieve $\pi_A \geq_m \pi_B$. For this she
produces some thorny blocks in chains $C_A$ and $C_B$ which will allow her to
claim blocks of chain $C_B$ as if they were of chain $C_A$ in her suffix proof.

The general form of this attack for an adversary sewing blocks to one forked
chain is illustrated in Figure \ref{fig:generic_attack}. Dashed arrows
represent interlink pointers of some level $\mu_A$. Starting from a thorny
block in the adversary's forked chain and following the interlink pointers,
a chain is formed which consists the adversary's suffix proof. Blocks of 
both chains are included in this proof and a verifier could not distinguish 
the non-smooth pointers participating in this proof chain and, as a result,
would consider it a valid proof.

\begin{figure}[h]
	\begin{center}
		\includegraphics[scale=0.75]{figures/generic_chainsewing_attack.pdf}
	\end{center}
	\caption{\textit{Generic Chainsewing Attack. $C_B$ is the chain of an honest
	player and $C_A$ the adversary's chain. Adversarially generated blocks are
	colored black. Dashed arrows represent interlink pointers included in the 
	adversary's suffix proof. Wavy lines imply one or more blocks.}}
	\label{fig:generic_attack}
\end{figure}

As the generic attack scheme may seem a bit complicated we will now describe a more
specific attack case. Consider that the adversary acts as described below.
Assume that the adversary chooses to attack at some level $\mu_A$. As shown in
Figure \ref{fig:attack} she first generates a superblock $b'$ in her forked chain
$C_A$ and a thorny block $a'$ in the honest chain $C_B$ which points to $b'$. 
As argued earlier, block $a'$ will be accepted as valid in the honest chain $C_B$ 
despite the invalid interlink pointers. After that, the adversary may mine on chain
$C_A$ or $C_B$, or not mine at all. At some point she produces a thorny block $a$ 
in $C_A$ pointing to a block $b$ of $C_B$. Because of the way blocks are generated
by updated honest miners there will be successive interlink pointers leading
from block $b$ to block $a'$. Thus following the interlink pointers a chain is
formulated which connects $C_A$ blocks $a$ and $b'$ and contains an arbitrarily
large part of the honest player's chain $C_B$.

At this point the adversary will produce a suffix proof for chain $C_A$ containing
the subchain $C\{ab\} \cup C\{b:a'\} \cup C\{a':b'\}$. Notice that following the
interlink pointers constructed in such a way, a light client perceives $C\{ab\}
\cup C\{b:a'\} \cup C\{a':b'\}$  as a valid chain.

\begin{figure}[h!]
	\begin{center}
		\includegraphics[scale=0.55]{figures/chainsewing_attack.png}
	\end{center}
	\caption{\textit{Chainsewing Attack. $C_B$ represents the chain of an honest
	player. $C_A$ is an adversarial fork. Adversarially generated blocks are colored
	black. Dashed arrows represent	interlink pointers included in the adversary's
	suffix proof. Wavy lines imply one or more blocks. Firm lines imply the previousId 
	relationship between two sequential blocks.}}
	\label{fig:attack}
\end{figure}

In this attack the adversary uses thorny blocks to ``sew`` portions of
the chain adopted by an honest player to her own forked chain. This remark 
justifies the name given to the attack.

Note that in order to make this attack successful, the adversary has to produce
only a few superblocks which let her arrogate an arbitrarily large number of blocks. 
Thus this attack is expected to succeed with overwhelming probability.

@TODO

Needs to be proven. It is not obvious that the attacker will succeed in high
probability, since the most important adversarially generated blocks , $a$
and $a'$, set a limit to the adversarial blocks produced in parallel to the
honest blocks of subchain $C\{ab\} \cup C\{b:a'\} \cup C\{a':b'\}$ and can
take part in the suffix proof.
