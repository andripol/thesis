\subsection{Velvet Interlinks}
More recently, velvet forks have been introduced~\cite{velvet}. In a
velvet fork, blocks created by upgraded miners (called \emph{velvet blocks}) are
accepted by unupgraded miners as in a soft fork. Additionally, blocks created by
unupgraded miners are also accepted by upgraded miners. This allows the protocol
to upgrade even if only a minority of miners chooses to upgrade. To maintain
backwards compatibility and to avoid causing forks, the additional data included
in a block is \emph{advisory} and must be accepted whether it exists or not.
Even if the additional data is invalid or malicious, upgraded nodes (in this
context also called \emph{velvet nodes}) are forced to accept the blocks. The
simplest approach to velvet fork the chain for interlinking purposes is to have
upgraded miners include the interlink pointer in the blocks they produce, but
accept blocks with missing or incorrect interlinks. As we show in the next
section, this approach is flawed and susceptible to unexpected attacks. A
surgical change in the way velvet blocks are produced is necessary to achieve
proper security.

In a velvet fork, only a minority of honest parties needs to support the protocol
changes. We refer to this percentage as the ``velvet parameter''.

\begin{definition}[Velvet Parameter]
	The \emph{velvet parameter} $g$ is defined as the percentage of honest parties
	that have upgraded to the new protocol. The absolute number of honest upgraded
	parties is denoted $n_h$ and it holds that
	$n_h = g (n - t)$.
	\label{defn:velvet_honest_majority}
\end{definition}

Velvet forks maintain backwards and forwards compatibility. This requires any
block produced by upgraded miners to be accepted by unupgraded nodes (as in a
soft fork), but also blocks produced by unupgraded miners to be accepted by
upgraded nodes. For the particular case of superblock NIPoPoWs under velvet
forks, upgraded miners must include the interlink data structure within their
blocks, but must also accept blocks missing the interlink structure or
Velvet forks maintain backwards and forwards compatibility. This requires any
block produced by upgraded miners to be accepted by unupgraded nodes (as in a
soft fork), but also blocks produced by unupgraded miners to be accepted by
upgraded nodes.
containing an invalid interlink. Unupgraded honest nodes will produce blocks
that contain no interlink, while upgraded honest nodes will produce blocks that
contain truthful interlinks. Therefore, any block with invalid interlinks will
be adversarially generated. However, such blocks cannot be rejected by the
upgraded nodes, as that would give the adversary an opportunity to cause a hard
fork.

A block generated by the adversary can thus contain arbitrary data in the
interlink and yet be adopted by an honest party. Because the honest prover is an
upgraded full node, it can determine what the correct interlink pointers are by
examining the whole previous chain, and can thus deduce whether a block contains
invalid interlink data. In that case, the prover can simply treat such blocks as
unupgraded. In the context of the attack that will be presented in the following
section, we examine the case where the adversary includes false interlink
pointers.

In any velvet protocol, a specific portion within a block, which is treated as
a comment by unupgraded nodes, is reused to contain auxiliary data by upgraded
miners. Because these auxiliary data can be deterministically calculated,
upgraded full nodes can verify the authenticity of the data in a new block they
receive. We distinguish blocks based on whether they follow the velvet protocol
rules or they deviate from them.

\begin{definition}[Smooth and Thorny blocks]
A block in a velvet protocol upgrade is called \emph{smooth} if it contains
auxiliary data and the data corresponds to the honest upgraded protocol. A block
is called \emph{thorny} if it contains auxiliary data, but the data differs from
the honest upgraded protocol. A block can be neither smooth nor thorny if it
does not contain auxiliary data.
\end{definition}

In the case of velvet forks for interlink purposes, the auxiliary data consists
of the Merkle Tree containing the interlink pointers to the most recent
superblock ancestor at every level $\mu$.