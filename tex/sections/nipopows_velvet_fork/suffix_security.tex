Our security proof is based on the security proof of NIPoPoWs Suffix Security Proof
under a soft or hard fork. Keep in mind that we operate under the condition of
(1/3)-bounded adversary.\\

The notion of Chain Quality is decisive for the operation of NIPoPoWs under a
velvet fork. Before stepping into the security proof we provide the formal
definition of this notion and the basic Theorem where its basic metric is computed.\\

\textbf{Definition 3 }(Chain Quality Property)\cite{Backbone}. The chain quality
property $Q_{cq}$ with parameters $\mu_{cq} \in \mathbb{R}$ and $l \in \mathbb{N}$
states that for any honest party $P$ with chain $C$ it holds that for any
$l $ consecutive blocks of $C$ the ratio of honest blocks is at least $\mu_{cq}$.\\

\textbf{Theorem 3 }(Chain Quality)\cite{Backbone}. \textit{In a typical execution
the chain quality property holds with parameter $\mu_{cq} > 1 - (1 + \frac{\delta}{2})
\cdot \frac{t}{n-t} - \frac{\delta}{2}$.}\\
\textit{Proof.} See \cite{Backbone}.\\

Our interest lies in keeping $\mu_{cq} \geq \dfrac{1}{2}$.
From Theorem 3 we have:
\begin{equation*}
	1 - (1 + \frac{\delta}{2}) \cdot \frac{t}{n-t} - \frac{\delta}{2} \geq \dfrac{1}{2} \Rightarrow 
	(1 + \delta)t \leq (1-\delta)(n-t)
\end{equation*}

Thus we have:
\begin{center}
\begin{equation}
	t \leq \dfrac{1 - \delta}{3}\cdot n
\end{equation}
\end{center}


\begin{thm}{\textbf{Suffix Proofs Security under velvet fork}}
	Assuming honest majority under velvet fork conditions such that $t \leq \dfrac{1 - \delta}{3}\cdot n_v$, the non-interactive proofs-of-proof-of-work construction for computable $\kappa$-stable monotonic suffix-sensitive predicates under velvet fork conditions is secure with overwhelming probability in $\kappa$.
\end{thm}
\textit{Proof.} By contradiction. We follow the proof construction of Theorem 2
and extend it. Let $Q$ be a $\kappa-$stable monotonic suffix-sensitive chain
predicate. Assume NIPoPoWs under velvet fork on $Q$ is insecure. Then, during
an execution at some round  $r_3$, $Q(C)$ is defined and the verifier $V$
disagrees with some honest participant. Assume the execution is typical. $V$
communicates with adversary $A$ and honest prover $B$. The verifier receives
proofs $\pi_A, \pi_B$. Because $B$ is honest, $\pi_B$ is a proof constructed
based on underlying blockchain $C_B$ (with $\pi_B \subseteq C_B$), which $B$
has adopted during round $r_3$ at which $\pi_B$ was generated. Consider $C_A$
the chain containing at least some of the blocks in $\pi_A$, while the remaining
blocks $\pi_A$ must belong in $C_B$.
The verifier outputs $\neg Q(C_B)$. Thus it is necessary that $\pi_A \geq \pi_B$.
We show that $\pi_A \geq \pi_B$ is a negligible event.
Let $b = LCA(\pi_A, \pi_B)$. Let the levels of comparison decided by the verifier
be $\mu_A$ and $\mu_B$ respectively. Let $\mu'_B$ be the adequate level of proof
$\pi_B$  with respect to block $b$. Call $\alpha_A = \pi_A \uparrow^{\mu_A}\{b:\}$,
$\alpha'_B = \pi_B \uparrow^{\mu'_B}\{b:\}$.

%%%%%%%%   reconsider this paragraph
Our proof construction is based on the following scheme: we show that the competing
suffix proofs can be conceived as consisting of three distinct parts. Each part
denotes a specific round set and is called after the number of blocks existing
in $\pi_A$ for that round set. Part $k_1$ lies for the first part of the proofs
between blocks $b = LCA(\pi_A, \pi_B)$ and $b_2 = LCA(C_A, C_B)$ meaning for the
common 0-level part of $\alpha_A,  \alpha_B$. Part $k_2$ lies for the second part
of the proofs considering the rounds from block $b_2 = LCA(C_A, C_B)$ up until
the Common Prefix is established at the 0-level chains for that fork point.
The third and last part, $k_3$ lies for the rest blocks in the proofs.\\
The above are illustrated, among other, in Parts I, II of Figure \ref{fig:proof_velvet}.

\begin{figure}[h!]
	\begin{center}
		\includegraphics[scale=0.5]{figures/proof_velvet.png}
	\end{center}
	\caption{\textit{ Wavy lines imply one or more blocks. Dashed lines and arrows imply
	interlink pointers to superblocks. \textbf{I}: the three round sets in two competing
	proofs at different levels, \textbf{II}: the corresponding 0-level chains,
	\textbf{III}: blocks participating in chains $C_B$, $C_A$ as conceived by 
    the verifier's perspective.}}	
    \label{fig:proof_velvet}
\end{figure}

We will now show three successive claims under velvet fork conditions: First,
$\alpha_A \downarrow \uparrow^{\mu_A}$ and $\alpha'_B \downarrow$ are mostly
disjoint. Second, $a_A$ contains mostly adversarially generated blocks. And third,
the adversary is able to produce this $a_A$ with negligible probability.\\
Let $\alpha_A = k_1 + k_2 + k_3$ and let $k_1, k_2, k_3$ be as defined in the
following Claims.\\
Let round $r_1$ be the round when block $b$ is generated and round $r_2$ when block
$b_2 = LCA(\alpha_A, \alpha'_B\downarrow)$ is generated.\\

\textbf{Claim 1:} As for honestly generated blocks, $\alpha_A$ and
$\alpha'_B\downarrow$ are mostly disjoint. Following the proof of Theorem 2
we conclude that $\vert \alpha_A\downarrow\uparrow^{\mu_A}[1:] \cap
\alpha'_B\downarrow[1:] \vert \leq k_{1} = 2^{\mu'_B - \mu_A}$. In order to see
this under the velvet fork conditions, first consider that the adversary behaves
honestly for blocks in her proof between $b$ and $b_2$. In this case Claim 1 of
Theorem 2 applies directly. In the opposite case, the adversary includes a block
with false interlink after block $b$ and before block $b_2$, thus the inequality
holds and because of Lemma 5 no more honestly generated blocks can be included
in $\alpha_A$ and we can immediately proceed to Claim 3 of this proof.

So we conclude that there are at least $\vert \alpha_A \vert - k_1$ blocks after
block $b$ in $\alpha_A$ which are not honestly generated blocks existing in
$\alpha'_B\downarrow$. In other words, there are $\vert \alpha_A \vert - k_1$
blocks after block $b$ in $\alpha_A$, which are either adversarially generated
existing in $\alpha_B\downarrow$ either don't belong in $\alpha_B\downarrow$.
This makes $b_2$ the last block before the 0-level fork point included in the
adversary's proof.\\

\textbf{Claim 2.} 
At least $k_3$ superblocks of $\alpha_A$ are adversarially generated. Just as
the proof of Theorem 2 and using a similar notation, because of the Common Prefix
property on parameter $k_{2\downarrow}$, $\alpha_A[k_{1}+k_{2}:]$ could contain
no honestly generated blocks. In order to see this for the velvet fork conditions
let's again consider the case that the adversary behaves honestly for the first
$(k_1 + k_2)$ blocks of her proof in which case Claim 2 of Theorem 2 is immediately
applied. In the opposite case, consider that the adversary includes in her proof a
block with invalid interlink at some earlier point. Again, because of Lemma 5 no more
honestly generated blocks can be included in $\alpha_A$ and we can proceed to Claim 3
of this proof.\\

%\begin{figure}[h!]
%	\begin{center}
%		\includegraphics[scale=0.5]{figures/exclude.png}
%	\end{center}
%	\caption{\textit{ Wavy lines imply one or more blocks. Dashed arrows imply interlink 
%pointers to superblocks. Adversarially generated blocks are colored black. Grey colored 
%blocks may be honestly or adversarially generated. \textbf{I}: the 0-level chains, 
%\textbf{II}: the corresponding proof chains; some blocks generated in $C_A$ are excluded 
%from proof $\pi_A$ in favor of the sewed blocks from $C_B$.}}
%	\label{fig:exclude}
%\end{figure}

\textbf{Claim 3.} Adversary can submit a suffix proof such that $\alpha_A >= \alpha_B$
with negligible probability.
%%% @TODO: make formal arguments
The last $k_3$ blocks included in $\alpha_A$ may belong either in $C_A$ either in
$C_B$ but are all adversarially generated. In the worst case  all $k_3 $ blocks
are sewed from $C_B$. This is the worst case scenario since each adversarially
generated block in $C_B$ may have dropped one honest block out of the chain
because of selfish mining. Considering this scenario, because of the strengthened
Honest Majority Assumption for $(1/3)$-bounded adversary, Theorem 3 for Chain
Quality guarantees that the majority of the blocks in $C_B$ was computed by
honest parties, thus the honestly generated blocks in $C_B$ for the same round
set sum to more amount of hashing power.\\
From all the above Claims we have that:\\
In the first round set, because of the common underlying chain:
\begin{equation} \label{eq_v_round_set_1}
2^{\mu_A} \vert \alpha_A^{k_1} \vert \leq 2^{\mu'_B} \vert \alpha'{_B^{k_1}} \vert
\end{equation}
Because of the adoption by an honest party of chain $C_B$ at a later round $r_3$, we
have for the second round set:
\begin{equation} \label{eq_v_round_set_2}
2^{\mu_A} \vert \alpha_A^{k_2} \vert \leq 2^{\mu'_B} \vert \alpha'{_B^{k_2}} \vert
\end{equation}
In the third round set, because of good Chain Quality under the strengthened Honest
Majority Assumption and Theorem 3 we have:
\begin{equation} \label{eq_v_round_set_3}
2^{\mu_A} \vert \alpha_A^{k_3} \vert < 2^{\mu'_B} \vert \alpha'{_B^{k_3}} \vert
\end{equation}
Consequently we have:

\begin{equation} \label{eq_v_all_round_sets}
2^{\mu_A} \vert \alpha_A \vert < 2^{\mu'_B} \vert \alpha'{_B} \vert
\end{equation}
