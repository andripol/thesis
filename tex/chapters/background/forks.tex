\section{Hard, Soft and Velvet Forks}
We typically describe the two common types of a blockchain permanent fork as follows.

A \textit{hard fork} is a consensus protocol upgrade which is not backwards
compatible. This means that the changes in the protocol break the old rules
since the block header's contents change. After a hard fork blocks generated
by upgraded players are not accepted by the unupgraded ones. In order the
protocol update to be well established, the majority of the players must be
upgraded at an early point or else the non-upgraded players may maintain the
longest chain under the old rules.

A \textit{soft fork} is a consensus protocol upgrade which is backwards compatible.
This is usually implemented by keeping the old rules while adding additional
information in a way that unupgraded players can ignore as comments, for example,
by adding adding data in the coinbase transaction. In this way unupgraded players
accept blocks generated by upgraded miners as valid, while, typically, unupgraded
blocks are not accepted by upgraded players. Players are motivated to upgrade in
order their blocks to be accepted in the chain as valid.

A \textit{velvet fork} is also a backwards compatible consensus protocol upgrade.
Similar to soft fork additional data can be inserted in the coinbase transaction.
A velvet fork requires any block compliant to the old protocol rules only to be
accepted as valid by both unupgraded and upgraded players. By requiring upgraded
miners to accept all blocks, even if they contain false data according to the new
protocol rules, we do not modify the set of accepted blocks. Therefore, the upgrade
is rather a \textit{recommendation} and not an actual change of the consensus
protocol.  In reality, the blockchain is never forked. Only the codebase is
upgraded and the data on the blockchain is interpreted differently\cite{nipopows}.

The goal of this work is to provide a modified NIPoPoWs protocol so that it can be
deployed under a velvet fork in a provably secure manner.